%% Modelo de currículo desenvolvido por Raquel Pereira.
%% <rpa1tera at gmail dot com>

%Cabeçalho
\documentclass[a4paper,10pt]{report}
\usepackage[utf8]{inputenc}
\usepackage[brazilian]{babel}
\usepackage[top=1.8cm,left=1.6cm,right=1.6cm,bottom=1.8cm]{geometry}
\usepackage{microtype}
\usepackage[colorlinks=true,urlcolor=red]{hyperref}
\usepackage{relsize}

%Comandos
\newcommand{\cpp}{C\kern-.133em\raise.45ex\hbox{\smaller[3]{++}}}

\newcommand{\nome}{Raquel Pereira de Alcântara}



%Início do documento
\begin{document}
\pagestyle{empty}

%Inicia cabeçalho
\begin{center}
%Nome
\begin{LARGE} 
 \textbf{\nome}
\end{LARGE}

%Informações de contato
\vspace{4em}
\begin{footnotesize}
Rua 08, Qd 11, 06 \hfill Solteira, 27 anos%

Bairro Cardoso, Aparecida de Goiânia - GO \hfill E-mail: rpa1tera@gmail.com%


Telefones: (62) 99927-1090 / (62) 3258-2474 \hfill Carteira de habilitação (CNH): Tipo AB%
\end{footnotesize}

\vspace{-.7em}
\hrulefill

\end{center}
%Finaliza cabeçalho

%Apresentação
\vspace{1em}
\textbf{\large{Apresentação}}
\newline

Tenho como objetivo profissional colaborar no meu ambiente de trabalho onde possa colocar em prática 

meus conhecimentos em favor da organização na qual viso integrar, objetivando crescer profissionalmente, adquirir 

experiência e aprendizado através do meu desenvolvimento, para oferecer um bom trabalho, procurando sempre o 

crescimento da organização.

%Formação
\vspace{1em}
\textbf{\large{Formação}}

\begin{itemize}
\addtolength{\baselineskip}{-1\baselineskip}

 \item Finalizando trabalho de conclusão do curso Bacharelado em Sistemas de Informação - Instituto Federal de Educação, Ciência e Tecnologia de Goiás.

\end{itemize}

%Idiomas
\vspace{1em}
\textbf{\large{Idiomas}}

\begin{itemize}
\addtolength{\baselineskip}{-1\baselineskip}
\item Inglês -- Lê bem, compreende bem e escreve razoavelmente;

\end{itemize}

%Formação Complementar
\vspace{1em}
\textbf{\large{Formação complementar}}

\begin{itemize}
\addtolength{\baselineskip}{-1\baselineskip}

 \item Cleaning Data in Python Course - DataCamp/2017
 \item Python Fundamentos para Análise de Dados - Data Science Academy/2017
 \item Mineração e Regras de Associação com Python e SQL - Jones Granatyr/2018
 \item ADVPL I – Totvs GO;
 \item Planejamento e Controle de Produção – Totvs GO;
 \item Arquitetura e Instalação Protheus 12 - User Function Treinamentos/2017;
 \item Testes de Software Automáticos - Gustavo Farias/2017;
\end{itemize}

%Experiência Profissional
\vspace{1em}
\textbf{\large{Experiência profissional}}
\\

%\addtolength{\baselineskip}{-1\baselineskip}

\textbf{Dhosp Distribuidora Hospitalar Importação e Exportação}

Analista de TI

Principais Atividades:
\begin{itemize}
\item Análise, manutenção e desenvolvimento de soluções para o ERP Protheus.;
\item Foco nos módulos: Gestão de Pessoal, Estoque/Custos, Faturamento e WMS;
\item Virada de versão Protheus 11 para 12;
\end{itemize}

Período: 03/2017 até a  data atual.
\newline

\textbf{Zuppani Industrial LTDA}

Analista de TI

Principais Atividades:
\begin{itemize}
\item Análise, manutenção, desenvolvimento, suporte e treinamento a usuários nos módulos Compras,
Estoque/Custos, Financeiro, Gestão de Pessoal, Oms, Faturamento e Pcp;

\item Virada de versão Protheus 10 para 11;
\item Virada de versão Protheus 11 para 12;

\item Implantação do sistema de Help Desk – GLPI;

\item Implantação de solução de comunicação corporativa Openfire – Spark;

\item Implantação de solução Firewall/Proxy Endian Community.
\end{itemize}

Período: 01/2009 até 02/2017. 	



%Participação em Eventos
\vspace{1em}
\vspace{1em}
\textbf{\large{Participação em Eventos}}
\newline

Atuante na comunidade de software livre, colaborando com atividades em eventos, tais como: 
Fórum 

Goiano de Software Livre - FGSL, Festival Latino-americano de Instalação de Software Livre - FLISOL, Google 

Women Techmakers.
\newline

2017

\begin{itemize}
\addtolength{\baselineskip}{-1\baselineskip}

 \item Campus Party - CPBR10 - São Paulo/SP - 700 horas de conteúdo nas áreas de ciência, criatividade, empreendedorismo e inovação;

 \item RoadSec - Conferência de Segurança da Informação - Goiânia/GO;

 \item Fórum Goiano de Software Livre - Goiânia - Coordenadora do GT de Divulgação;

 \item Encontro Anual de Tecnologia da Informação do Oeste Goiano - Instituto Federal Goiano Iporá/GO - Palestrante;

 \item II Simpósio de Tecnologia da Informação - Instituto Federal Goiano Ceres - Palestrante;
 
 \item XIV Congresso Latino Americano de Software Livre e Tecnologias Abertas - Foz do Iguaçu/PR.
\end{itemize}

2018
\begin{itemize}
 \item Campus Party - CPBR11 - São Paulo/SP - 900 horas de conteúdo nas áreas de ciência, inovação, tecnologia, empreendedorismo, games e criatividade;
 \item 5º Encontro Women Techmakers - Goiânia/GO - evento promovido pelo Google e pelo grupo de desenvolvedores Google do estado de Goiás.  O Women Techmakers é um programa que visa incentivar a participação de mulheres na área de tecnologia - Coordenação.
 \item III Simpósio de Tecnologia da Informação - Instituto Federal Goiano Ceres - Palestrante.
\end{itemize}

\vfill
%Rodapé
\hrulefill

\begin{footnotesize}
\textsuperscript{$\ast$} Esse currículo, junto com seu código fonte em \LaTeX, pode ser encontrado em \href{https://github.com/rpa1tera/curriculo}{github.com/rpa1tera/curriculo}
\end{footnotesize}

\end{document}